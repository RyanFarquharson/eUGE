\documentclass[]{article}
\usepackage{lmodern}
\usepackage{amssymb,amsmath}
\usepackage{ifxetex,ifluatex}
\usepackage{fixltx2e} % provides \textsubscript
\ifnum 0\ifxetex 1\fi\ifluatex 1\fi=0 % if pdftex
  \usepackage[T1]{fontenc}
  \usepackage[utf8]{inputenc}
\else % if luatex or xelatex
  \ifxetex
    \usepackage{mathspec}
  \else
    \usepackage{fontspec}
  \fi
  \defaultfontfeatures{Ligatures=TeX,Scale=MatchLowercase}
\fi
% use upquote if available, for straight quotes in verbatim environments
\IfFileExists{upquote.sty}{\usepackage{upquote}}{}
% use microtype if available
\IfFileExists{microtype.sty}{%
\usepackage{microtype}
\UseMicrotypeSet[protrusion]{basicmath} % disable protrusion for tt fonts
}{}
\usepackage[margin=1in]{geometry}
\usepackage{hyperref}
\hypersetup{unicode=true,
            pdftitle={eUGE - electricity Use Generation and Export},
            pdfauthor={Ryan Farquharson},
            pdfborder={0 0 0},
            breaklinks=true}
\urlstyle{same}  % don't use monospace font for urls
\usepackage{color}
\usepackage{fancyvrb}
\newcommand{\VerbBar}{|}
\newcommand{\VERB}{\Verb[commandchars=\\\{\}]}
\DefineVerbatimEnvironment{Highlighting}{Verbatim}{commandchars=\\\{\}}
% Add ',fontsize=\small' for more characters per line
\usepackage{framed}
\definecolor{shadecolor}{RGB}{248,248,248}
\newenvironment{Shaded}{\begin{snugshade}}{\end{snugshade}}
\newcommand{\KeywordTok}[1]{\textcolor[rgb]{0.13,0.29,0.53}{\textbf{#1}}}
\newcommand{\DataTypeTok}[1]{\textcolor[rgb]{0.13,0.29,0.53}{#1}}
\newcommand{\DecValTok}[1]{\textcolor[rgb]{0.00,0.00,0.81}{#1}}
\newcommand{\BaseNTok}[1]{\textcolor[rgb]{0.00,0.00,0.81}{#1}}
\newcommand{\FloatTok}[1]{\textcolor[rgb]{0.00,0.00,0.81}{#1}}
\newcommand{\ConstantTok}[1]{\textcolor[rgb]{0.00,0.00,0.00}{#1}}
\newcommand{\CharTok}[1]{\textcolor[rgb]{0.31,0.60,0.02}{#1}}
\newcommand{\SpecialCharTok}[1]{\textcolor[rgb]{0.00,0.00,0.00}{#1}}
\newcommand{\StringTok}[1]{\textcolor[rgb]{0.31,0.60,0.02}{#1}}
\newcommand{\VerbatimStringTok}[1]{\textcolor[rgb]{0.31,0.60,0.02}{#1}}
\newcommand{\SpecialStringTok}[1]{\textcolor[rgb]{0.31,0.60,0.02}{#1}}
\newcommand{\ImportTok}[1]{#1}
\newcommand{\CommentTok}[1]{\textcolor[rgb]{0.56,0.35,0.01}{\textit{#1}}}
\newcommand{\DocumentationTok}[1]{\textcolor[rgb]{0.56,0.35,0.01}{\textbf{\textit{#1}}}}
\newcommand{\AnnotationTok}[1]{\textcolor[rgb]{0.56,0.35,0.01}{\textbf{\textit{#1}}}}
\newcommand{\CommentVarTok}[1]{\textcolor[rgb]{0.56,0.35,0.01}{\textbf{\textit{#1}}}}
\newcommand{\OtherTok}[1]{\textcolor[rgb]{0.56,0.35,0.01}{#1}}
\newcommand{\FunctionTok}[1]{\textcolor[rgb]{0.00,0.00,0.00}{#1}}
\newcommand{\VariableTok}[1]{\textcolor[rgb]{0.00,0.00,0.00}{#1}}
\newcommand{\ControlFlowTok}[1]{\textcolor[rgb]{0.13,0.29,0.53}{\textbf{#1}}}
\newcommand{\OperatorTok}[1]{\textcolor[rgb]{0.81,0.36,0.00}{\textbf{#1}}}
\newcommand{\BuiltInTok}[1]{#1}
\newcommand{\ExtensionTok}[1]{#1}
\newcommand{\PreprocessorTok}[1]{\textcolor[rgb]{0.56,0.35,0.01}{\textit{#1}}}
\newcommand{\AttributeTok}[1]{\textcolor[rgb]{0.77,0.63,0.00}{#1}}
\newcommand{\RegionMarkerTok}[1]{#1}
\newcommand{\InformationTok}[1]{\textcolor[rgb]{0.56,0.35,0.01}{\textbf{\textit{#1}}}}
\newcommand{\WarningTok}[1]{\textcolor[rgb]{0.56,0.35,0.01}{\textbf{\textit{#1}}}}
\newcommand{\AlertTok}[1]{\textcolor[rgb]{0.94,0.16,0.16}{#1}}
\newcommand{\ErrorTok}[1]{\textcolor[rgb]{0.64,0.00,0.00}{\textbf{#1}}}
\newcommand{\NormalTok}[1]{#1}
\usepackage{graphicx,grffile}
\makeatletter
\def\maxwidth{\ifdim\Gin@nat@width>\linewidth\linewidth\else\Gin@nat@width\fi}
\def\maxheight{\ifdim\Gin@nat@height>\textheight\textheight\else\Gin@nat@height\fi}
\makeatother
% Scale images if necessary, so that they will not overflow the page
% margins by default, and it is still possible to overwrite the defaults
% using explicit options in \includegraphics[width, height, ...]{}
\setkeys{Gin}{width=\maxwidth,height=\maxheight,keepaspectratio}
\IfFileExists{parskip.sty}{%
\usepackage{parskip}
}{% else
\setlength{\parindent}{0pt}
\setlength{\parskip}{6pt plus 2pt minus 1pt}
}
\setlength{\emergencystretch}{3em}  % prevent overfull lines
\providecommand{\tightlist}{%
  \setlength{\itemsep}{0pt}\setlength{\parskip}{0pt}}
\setcounter{secnumdepth}{0}
% Redefines (sub)paragraphs to behave more like sections
\ifx\paragraph\undefined\else
\let\oldparagraph\paragraph
\renewcommand{\paragraph}[1]{\oldparagraph{#1}\mbox{}}
\fi
\ifx\subparagraph\undefined\else
\let\oldsubparagraph\subparagraph
\renewcommand{\subparagraph}[1]{\oldsubparagraph{#1}\mbox{}}
\fi

%%% Use protect on footnotes to avoid problems with footnotes in titles
\let\rmarkdownfootnote\footnote%
\def\footnote{\protect\rmarkdownfootnote}

%%% Change title format to be more compact
\usepackage{titling}

% Create subtitle command for use in maketitle
\newcommand{\subtitle}[1]{
  \posttitle{
    \begin{center}\large#1\end{center}
    }
}

\setlength{\droptitle}{-2em}

  \title{eUGE - \textbf{e}lectricity \textbf{U}se \textbf{G}eneration and
\textbf{E}xport}
    \pretitle{\vspace{\droptitle}\centering\huge}
  \posttitle{\par}
    \author{Ryan Farquharson}
    \preauthor{\centering\large\emph}
  \postauthor{\par}
      \predate{\centering\large\emph}
  \postdate{\par}
    \date{9 August 2018}


\begin{document}
\maketitle

Being a closet greenie, and a scientist, I have been fascinated by
energy and water use. Many people decry the cost of utilities with
little to no understanding of how much they use, nor how much they
waste. An almost daily ritual for me is to wander up to the water meter,
then down to the rain gauge, and back up to the electricity meter and
record all the numbers, by hand, into an exercise book. I have around a
thousand entries. Until now, I have done little with that data.

This presented a great oportunity to hone some skills in data
exploration and visualisation, plus setting up file structures, version
control and doing some simple calcualtions etc. What is described here
is not necessarily the best way to do things. Let's say it's been a good
learning experience!

\subsubsection{Methods}\label{methods}

\paragraph{Meter readings:}\label{meter-readings}

Meter readings were recorded by hand at the end of each day, whenever
possible, from May 2009 to present. The reads recorded are:

\begin{itemize}
\tightlist
\item
  Date (yyyy-mm-dd)
\item
  Peak (kwh)
\item
  Offpeak (kwh)
\item
  PV feedin (kwh)
\item
  PV generation (kwh)
\item
  PV hours (hours)
\end{itemize}

Rainfall and mains water were also recorded, and climate data can be
obtained from BOM. But we have left that out for now and will focus on
electricity.

The raw meter readings were transcribed into a spreadsheet and saved as
the meterreads.csv file. For this exercise, instad of trying to
transcribe all of the data, I just took the full reads that were closest
to the end of each month.

Some initial plots were done which revealed a number of typographical
errors. These were corrected by going back to the exercise book and
checking the data. Unfortunately I did not save or commit the incorrect
file or the plots. You'll just have to trust me on that! It was great
having the original excercise book on hand. If this was someone else's
data, QC would have been tricky.

\paragraph{Directory structures and version
control}\label{directory-structures-and-version-control}

Being the first time I've worked like this, I manually set up a project
folder and subfolders for doc, data, results. I also set up a git
repository and remote master. I then opened up a new project in Rstudio.
Note that I first saved this markdown file in the doc folder but ran
into trouble with the directory structure. The code below should solve
this problem.

\begin{verbatim}
{r setup, message = FALSE, warning = FALSE, result = "hide"}

knitr::opts_chunk$set(echo = TRUE, message=FALSE, warning = FALSE)

knitr::opts_knit$set(root.dir = rprojroot::find_rstudio_root_file())
\end{verbatim}

\paragraph{Data processing}\label{data-processing}

First I loaded the tidyverse library and used read\_csv to bring the
meterreads.csv file in as a tibble.

\begin{Shaded}
\begin{Highlighting}[]
\KeywordTok{library}\NormalTok{(tidyverse)}

\NormalTok{meterreads <-}\StringTok{ }\KeywordTok{read_csv}\NormalTok{(}\StringTok{"data/meterreads.csv"}\NormalTok{)}
\end{Highlighting}
\end{Shaded}

\begin{Shaded}
\begin{Highlighting}[]
\NormalTok{meterreads}
\end{Highlighting}
\end{Shaded}

\begin{verbatim}
## # A tibble: 111 x 7
##    date       peak_reading offpeak_reading feedin_reading PV_reading
##    <date>            <int>           <int>          <int>      <int>
##  1 2009-05-30         1872            3195            888       1161
##  2 2009-06-27         2030            3663            906       1191
##  3 2009-07-30         2196            4200            932       1227
##  4 2009-08-30         2230            4543            978       1286
##  5 2009-09-29         2406            4865           1042       1366
##  6 2009-10-30         2500            5047           1137       1478
##  7 2009-11-29         2556            5047           1230       1590
##  8 2009-12-27         2651            5047           1323       1702
##  9 2010-01-31         2753            5090           1448       1858
## 10 2010-02-28         2870            5145           1536       1977
## # ... with 101 more rows, and 2 more variables: PVhours_reading <int>,
## #   water_reading <int>
\end{verbatim}

Note that with tidyverse, the date came in as datatype which came in
handy down the track.

Some exploratory data analysis was done using ggplot.

\begin{Shaded}
\begin{Highlighting}[]
\CommentTok{#plot power reads i.e. peak, offpeak and PV}

\KeywordTok{ggplot}\NormalTok{(meterreads, }\KeywordTok{aes}\NormalTok{(}\DataTypeTok{x =}\NormalTok{ date)) }\OperatorTok{+}\StringTok{ }
\StringTok{  }\KeywordTok{geom_line}\NormalTok{(}\KeywordTok{aes}\NormalTok{(}\DataTypeTok{y =}\NormalTok{ peak_reading, }\DataTypeTok{colour =} \StringTok{"peak"}\NormalTok{)) }\OperatorTok{+}
\StringTok{  }\KeywordTok{geom_line}\NormalTok{(}\KeywordTok{aes}\NormalTok{(}\DataTypeTok{y =}\NormalTok{ offpeak_reading, }\DataTypeTok{colour =} \StringTok{"offpeak"}\NormalTok{)) }\OperatorTok{+}
\StringTok{  }\KeywordTok{geom_line}\NormalTok{(}\KeywordTok{aes}\NormalTok{(}\DataTypeTok{y =}\NormalTok{ PV_reading, }\DataTypeTok{colour =} \StringTok{"PV"}\NormalTok{)) }\OperatorTok{+}
\StringTok{  }\KeywordTok{geom_line}\NormalTok{(}\KeywordTok{aes}\NormalTok{(}\DataTypeTok{y =}\NormalTok{ feedin_reading, }\DataTypeTok{colour =} \StringTok{"PV feed-in"}\NormalTok{)) }\OperatorTok{+}
\StringTok{  }\KeywordTok{labs}\NormalTok{(}\DataTypeTok{x =} \StringTok{"Date"}\NormalTok{, }\DataTypeTok{y =} \StringTok{"kwh"}\NormalTok{, }\DataTypeTok{title =} \StringTok{"Power reads"}\NormalTok{, }\DataTypeTok{colour =} \StringTok{"Read type"}\NormalTok{) }\OperatorTok{+}
\StringTok{  }\KeywordTok{scale_color_manual}\NormalTok{(}\DataTypeTok{values =} \KeywordTok{c}\NormalTok{(}\StringTok{"blue"}\NormalTok{,}\StringTok{"red"}\NormalTok{,}\StringTok{"green"}\NormalTok{,}\StringTok{"black"}\NormalTok{)) }\OperatorTok{+}
\StringTok{  }\KeywordTok{theme}\NormalTok{(}\DataTypeTok{panel.background =} \KeywordTok{element_rect}\NormalTok{(}\DataTypeTok{fill =} \StringTok{"white"}\NormalTok{),}
        \DataTypeTok{panel.border =} \KeywordTok{element_rect}\NormalTok{(}\DataTypeTok{fill =} \OtherTok{NA}\NormalTok{))}
\end{Highlighting}
\end{Shaded}

\includegraphics{eUGE_report_files/figure-latex/exploratory plots-1.pdf}

So far so good. You can see the meter reads going up over time with some
bumps along the way, and if you look closely enough, some different
slopes at different times.

But that hides most of the story. What we actually need is usage -
i.e.~the increment between each read.

The first step to get usage is to get the diffence in each meter read
from the previous reading. To do this, I used mutate() and lag().

\begin{Shaded}
\begin{Highlighting}[]
\CommentTok{# make a new tibble called 'usage' from 'meterreads' and}
\CommentTok{# add a column called peak_usage by using mutate and lag}
\CommentTok{# mutate makes a new column called peak_usage by starting with peak_reading and subtracting a value}
\CommentTok{# this value is also from the peak_reading column, but by using lag it is from the previous row.}

\NormalTok{usage <-}\StringTok{ }\NormalTok{meterreads }\OperatorTok\StringTok{ }
\StringTok{  }\KeywordTok{mutate}\NormalTok{(}\DataTypeTok{peak_usage =}\NormalTok{ peak_reading }\OperatorTok{-}\StringTok{ }
\StringTok{           }\KeywordTok{lag}\NormalTok{(peak_reading, }\DataTypeTok{default =} \KeywordTok{first}\NormalTok{(peak_reading)))}

\CommentTok{# now repeat the process for the other readings}

\NormalTok{usage <-}\StringTok{ }\NormalTok{usage }\OperatorTok\StringTok{ }
\StringTok{  }\KeywordTok{mutate}\NormalTok{(}\DataTypeTok{offpeak_usage =}\NormalTok{ offpeak_reading }\OperatorTok{-}
\StringTok{           }\KeywordTok{lag}\NormalTok{(offpeak_reading, }\DataTypeTok{default =} \KeywordTok{first}\NormalTok{(offpeak_reading)))}

\NormalTok{usage <-}\StringTok{ }\NormalTok{usage }\OperatorTok\StringTok{ }
\StringTok{  }\KeywordTok{mutate}\NormalTok{(}\DataTypeTok{feedin_usage =}\NormalTok{ feedin_reading }\OperatorTok{-}
\StringTok{           }\KeywordTok{lag}\NormalTok{(feedin_reading, }\DataTypeTok{default =} \KeywordTok{first}\NormalTok{(feedin_reading)))}

\NormalTok{usage <-}\StringTok{ }\NormalTok{usage }\OperatorTok\StringTok{ }
\StringTok{  }\KeywordTok{mutate}\NormalTok{(}\DataTypeTok{PV_usage =}\NormalTok{ PV_reading }\OperatorTok{-}
\StringTok{           }\KeywordTok{lag}\NormalTok{(PV_reading, }\DataTypeTok{default =} \KeywordTok{first}\NormalTok{(PV_reading)))}

\NormalTok{usage <-}\StringTok{ }\NormalTok{usage }\OperatorTok\StringTok{ }
\StringTok{  }\KeywordTok{mutate}\NormalTok{(}\DataTypeTok{PVhours_usage =}\NormalTok{ PVhours_reading }\OperatorTok{-}
\StringTok{           }\KeywordTok{lag}\NormalTok{(PVhours_reading, }\DataTypeTok{default =} \KeywordTok{first}\NormalTok{(PVhours_reading)))}

\NormalTok{usage <-}\StringTok{ }\NormalTok{usage }\OperatorTok\StringTok{ }
\StringTok{  }\KeywordTok{mutate}\NormalTok{(}\DataTypeTok{water_usage =}\NormalTok{ water_reading }\OperatorTok{-}
\StringTok{           }\KeywordTok{lag}\NormalTok{(water_reading, }\DataTypeTok{default =} \KeywordTok{first}\NormalTok{(water_reading)))}

\NormalTok{usage <-}\StringTok{ }\NormalTok{usage }\OperatorTok\StringTok{ }
\StringTok{  }\KeywordTok{mutate}\NormalTok{(}\DataTypeTok{days =} \KeywordTok{as.integer}\NormalTok{(date }\OperatorTok{-}
\StringTok{           }\KeywordTok{lag}\NormalTok{(date, }\DataTypeTok{default =} \KeywordTok{first}\NormalTok{(date))))}

\NormalTok{usage <-}\StringTok{ }\KeywordTok{slice}\NormalTok{(usage, }\OperatorTok{-}\DecValTok{1}\NormalTok{)}
\end{Highlighting}
\end{Shaded}

There was probably a quicker way to do this, but for now, it did the
trick. One problem though, it gives us a bunch of zeros in the first
row. ggplot is smart enough to ignore these zeros, but really I should
clean up this data before moving on. See the last line where I take a
slice of usage except for the first row.

So we now have the increment of power usage, generation and export
between meter reads.

But guess what. Months have different numbers of days, and I couldn't
always do a meter read at the end of the month. So to compare between
months, we need to get usage per day.

This was done by dividing each usage by the number of days in each
increment. Happily because we're playing in the tidyverse, calculating
the number of days was a cinch. If you were paying attention, you'd have
seen it in the previous chunk.

\begin{Shaded}
\begin{Highlighting}[]
\CommentTok{# make a new table called perday.  divide usage by number of days to get usage per day.}

\NormalTok{perday <-}\StringTok{ }\NormalTok{usage }\OperatorTok\StringTok{ }
\StringTok{  }\KeywordTok{mutate}\NormalTok{(}\DataTypeTok{peak_perday =}\NormalTok{ peak_usage }\OperatorTok{/}\StringTok{ }\NormalTok{days)}

\NormalTok{perday <-}\StringTok{ }\NormalTok{perday }\OperatorTok\StringTok{ }
\StringTok{  }\KeywordTok{mutate}\NormalTok{(}\DataTypeTok{offpeak_perday =}\NormalTok{ offpeak_usage }\OperatorTok{/}\StringTok{ }\NormalTok{days)}

\NormalTok{perday <-}\StringTok{ }\NormalTok{perday }\OperatorTok\StringTok{ }
\StringTok{  }\KeywordTok{mutate}\NormalTok{(}\DataTypeTok{totalusage_perday =}\NormalTok{ peak_perday }\OperatorTok{+}\StringTok{ }\NormalTok{offpeak_perday)}

\NormalTok{perday <-}\StringTok{ }\NormalTok{perday }\OperatorTok\StringTok{ }
\StringTok{  }\KeywordTok{mutate}\NormalTok{(}\DataTypeTok{feedin_perday =}\NormalTok{ feedin_usage }\OperatorTok{/}\StringTok{ }\NormalTok{days)}

\NormalTok{perday <-}\StringTok{ }\NormalTok{perday }\OperatorTok\StringTok{ }
\StringTok{  }\KeywordTok{mutate}\NormalTok{(}\DataTypeTok{PVgen_perday =}\NormalTok{ PV_usage }\OperatorTok{/}\StringTok{ }\NormalTok{days)}

\NormalTok{perday <-}\StringTok{ }\NormalTok{perday }\OperatorTok\StringTok{ }
\StringTok{  }\KeywordTok{mutate}\NormalTok{(}\DataTypeTok{PVhours_perday =}\NormalTok{ PVhours_usage }\OperatorTok{/}\StringTok{ }\NormalTok{days)}

\NormalTok{perday <-}\StringTok{ }\NormalTok{perday }\OperatorTok\StringTok{ }
\StringTok{  }\KeywordTok{mutate}\NormalTok{(}\DataTypeTok{water_perday =}\NormalTok{ water_usage }\OperatorTok{/}\StringTok{ }\NormalTok{days)}

\NormalTok{perday <-}\StringTok{ }\NormalTok{perday }\OperatorTok\StringTok{ }
\StringTok{  }\KeywordTok{mutate}\NormalTok{(}\DataTypeTok{PVefficiency =}\NormalTok{ PVgen_perday }\OperatorTok{/}\StringTok{ }\NormalTok{PVhours_perday)}

\NormalTok{perday <-}\StringTok{ }\NormalTok{perday }\OperatorTok\StringTok{ }
\StringTok{  }\KeywordTok{mutate}\NormalTok{(}\DataTypeTok{feedinefficiency =}\NormalTok{ feedin_perday }\OperatorTok{/}\StringTok{ }\NormalTok{PVhours_perday)}
\end{Highlighting}
\end{Shaded}

Ok. Now we have something meaningful to work with.

This is about to get exciting! Having recored all of this data over 9
years, we're about to see what it looks like.

It took a few iterations and searching through help files to get to this
point, but here it is.

First up, some PV efficiency plots

\includegraphics{eUGE_report_files/figure-latex/efficiency plots-1.pdf}

Remember, this is a 1 kW PV system and in summer we're getting around
0.3 kW per hour. Winter looks depressing, doesn't it? Wondering what
that dip in summer is? Read on.

Now let's put everything together. Get ready for it\ldots{}

\includegraphics{eUGE_report_files/figure-latex/visualisation using ggplot-1.pdf}

So many stories to tell!

\subsubsection{So what's going on?}\label{so-whats-going-on}

Well, there's a bit of background to know.

When this kicked off it was just the two of us in a small weatherboard
cottage with a newish 1 kW PV system that we paid a lot of money for
(could get one 5 times bigger for the same dosh today).

We run entirely on electricity. We have a flat panel solar hot water
system with offpeak electric resistance backup. We also have a heat bank
which again runs on offpeak electricity. For a long time, this was our
sole source of heating.

Then in July 2011 bubs 1 arrived, followed shortly by a reverse cycle
heat pump in the living room. We've since been tweaking our mix of heat
bank (offpeak) and heat pump (peak).

In January 2014 bubs 2 arrived.

We also have visitors periodically when a supplementary hot water system
kicks in, running on off peak.

So with that info, you can start to interpret the data.

One other hint. Our PV panels point north. Around summer solstice, the
sun rises and sets in the south east and south west, so there are many
hours when the sun is up, the PVs are producing, but it's from diffuse
light rather than direct light. That's the dip in the efficiency plots
that you saw before.

\subsubsection{Key findings}\label{key-findings}

\paragraph{Seasonality}\label{seasonality}

Both energy use and generation in our household are highly seasonal.

Use peaks in winter.

Generation peaks in summer.

\paragraph{4 person households - how we
compare}\label{person-households---how-we-compare}

Depsite living in a leaky 1949 weatherboard home run entirely on
electricity, our total usage is less than the average local 4 person
household. Can we do better? Hell yeah. Stay tuned. We're currently
designing our energy efficient dream home.

\paragraph{Take home messages}\label{take-home-messages}

Unfortunately our electricity use and PV generation are completely out
of sync. For now, that's ok because we get a generous feed-in tarrif.
Credits in summer help pay some of the winter bills. But we're not
expecting that to last long.

It looks like taking the current house off grid would be a money sucking
challenge and counterproductive. We would need a big system to meet
heating demand in winter, and then will have a massive surplus in summer
with no-one to give it to. Given that our main electricity use is
heating, alternate fuels could be considered. Gas is avaiable but is a
dirty fossil fuel. Wood could be used, but it costs and requires work.

\subsubsection{Next steps}\label{next-steps}

It would be interesting to get daily data, pull in some climate data
from the BOM and pull out some relationships.

We can do the same for water usage.

The modeller in me wants to go crazy with it. Plate is too full.

As we plan our new energy efficient house, we are considering heating
options. Do we need to fork out big \$ for a very efficient, effective
and somewhat luxurious in-slab hydronic heating system? Do we suffice
with a wood oven? What about salvaging our heat pump and heat bank be
enough? Will we even need heating at all?

Whatever we end up with, rest assured that I will continue to record our
meter readings and get back to you with some interesting data. Hopefully
it won't take me another 9 years!


\end{document}
